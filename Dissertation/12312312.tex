\chapter{Регрессионный анализ многомерных временных рядов на основе динамической спайковой модели} \label{chapt2}

\section{Формальная постановка задачи} \label{sect2_1}
\indent Рассмотрим пару многомерных временных рядов $x(t) \in \mathbb{R}^{n_{x}}, y(t) \in \mathbb{R}^{n_{y}}$, где $n_{x}, n_{y}$ размерности $x(t), y(t)$, соответственно. Смоделировать временной ряд $y(t)$ на основе ряда $x(t)$, означает найти  такую динамическую систему 
\begin{equation}
\frac{dy'}{dt} = F(y'(t), x(t)),
\end{equation}
что интегральная кривая $y'(t)$ динамической системы удовлетворяет условию
\begin{equation}
C_{int} = \int_{0}^{T} C(y(t), y'(t)) dt \rightarrow 0,
\end{equation}
где $C(y'(t), y(t))$ функция ошибки обозначающая степень сходимости поиска модели. Поставим задачу моделирования временного ряда $y(t)$ на основе временного ряда $x(t)$ как задачу регресионного анализа.


\section{Модель динамической спайковой нейронной сети} \label{sect2_2}
\indent В качестве базовой модели для решения поставленной задачи рассмотрим пороговый интегратор (\textit{Integrate and fire} \cite{burkitt2006review}). Уравнение динамики задается системой ДУ
\begin{equation}
\tau_{mem} \frac{du}{dt} = -u(t) + w I(t)
\end{equation}
где $u(t) \in \mathbb{R}^{n}$, мембрана нейронной популяции размера $n$, $I(t) \in \mathbb{R}^{m}$ приложенный ток, рассматриваемый как входной сигнал приложенный к системе с целью получить отклик $u(t)$. $w$ -- матрица $n x m$, обозначающая веса нейронной популяции, которые формируют линейное преобразование приложенного тока $I(t)$. Отклик нейронной популяции $u(t)$ проходит нелинейное преобразование функцией активации
\begin{equation}
P_{act}(t) = F_{act}(u(t)),
\end{equation}
здесь $P_{act}(t)$ обозначает частоту негомогенного Пуассоновского процесса, реализация которого характеризует появления событий-спайков в нейронной популяции, которые можно представить в виду импульсной формы сигнала, или сумм дельта-функций
\begin{equation}
S(t) = \sum_{i=1}^n \sum_{f_{i}} \delta(t - t_{j}^{f_{i}}).
\end{equation}
где $f_{i} = {t_{1}^{f_{i}}, t_{2}^{f_{i}}, ..., t_{k}^{f_{i}}}$ множество всех произведенных спайков $i$-ым нейроном из популяции.
Рассмотрим частный случай где размерность исходного временного ряда $x(t)$ и размера популяции совпадают. Таким образом можно сказать, что задача минимизации значения функции ошибки $C_{int}$ означает нахождение 


\section{Обучение модели} \label{sect2_3}
