%%% Основные сведения %%%
\newcommand{\thesisAuthor}             % Диссертация, ФИО автора
{%
    \texorpdfstring{% \texorpdfstring takes two arguments and uses the first for (La)TeX and the second for pdf
        Чернышев Алексей Сергеевич% так будет отображаться на титульном листе или в тексте, где будет использоваться переменная
    }{%
        Чернышев, Алексей Сергеевич% эта запись для свойств pdf-файла. В таком виде, если pdf будет обработан программами для сбора библиографических сведений, будет правильно представлена фамилия.
    }%
}
\newcommand{\thesisAuthorShort}        % Диссертация, ФИО автора инициалами
{А.С.~Чернышев}

\newcommand{\thesisUdk}                % Диссертация, УДК
{004.896}
\newcommand{\thesisTitle}              % Диссертация, название
{\texorpdfstring{\MakeUppercase{Спайковая нейросетевая модель и ее применения}}{Название диссертационной работы}}
\newcommand{\thesisSpecialtyNumber}    % Диссертация, специальность, номер
{\texorpdfstring{05.13.18}{05.13.18}}
\newcommand{\thesisSpecialtyTitle}     % Диссертация, специальность, название
{\texorpdfstring{Математическое моделирование, численные методы и
комплексы  программ}{Математическое моделирование, численные методы и
комплексы  программ}}
\newcommand{\thesisDegree}             % Диссертация, ученая степень
{кандидата физико-математических наук}
\newcommand{\thesisDegreeShort}        % Диссертация, ученая степень, краткая запись
{канд. физ.-мат. наук}
\newcommand{\thesisCity}               % Диссертация, город защиты
{Москва}
\newcommand{\thesisYear}               % Диссертация, год защиты
{2017}
\newcommand{\thesisOrganization}       % Диссертация, организация
{Московский Государственный Технический Университет им. Н.Э. Баумана}
\newcommand{\thesisOrganizationShort}  % Диссертация, краткое название организации для доклада
{НазУчДисРаб}

\newcommand{\thesisInOrganization}     % Диссертация, организация в предложном падеже: Работа выполнена в ...
{Московском Государственном Техническом Университете им. Н.Э. Баумана}

\newcommand{\supervisorFio}            % Научный руководитель, ФИО
{Карпенко Анатолий Павлович}
\newcommand{\supervisorRegalia}        % Научный руководитель, регалии
{доктор физико-математических наук}
\newcommand{\supervisorFioShort}       % Научный руководитель, ФИО
{А.П.~Карпенко}
\newcommand{\supervisorRegaliaShort}   % Научный руководитель, регалии
{д. ф.-м. н.}


\newcommand{\opponentOneFio}           % Оппонент 1, ФИО
{Фамилия Имя Отчество}
\newcommand{\opponentOneRegalia}       % Оппонент 1, регалии
{доктор физико-математических наук, профессор}
\newcommand{\opponentOneJobPlace}      % Оппонент 1, место работы
{Не очень длинное название для места работы}
\newcommand{\opponentOneJobPost}       % Оппонент 1, должность
{старший научный сотрудник}

\newcommand{\opponentTwoFio}           % Оппонент 2, ФИО
{Фамилия Имя Отчество}
\newcommand{\opponentTwoRegalia}       % Оппонент 2, регалии
{кандидат физико-математических наук}
\newcommand{\opponentTwoJobPlace}      % Оппонент 2, место работы
{Основное место работы c длинным длинным длинным длинным названием}
\newcommand{\opponentTwoJobPost}       % Оппонент 2, должность
{старший научный сотрудник}

\newcommand{\leadingOrganizationTitle} % Ведущая организация, дополнительные строки
{Федеральное государственное бюджетное образовательное учреждение высшего профессионального образования с~длинным длинным длинным длинным названием}

\newcommand{\defenseDate}              % Защита, дата
{DD mmmmmmmm YYYY~г.~в~XX часов}
\newcommand{\defenseCouncilNumber}     % Защита, номер диссертационного совета
{Д\,123.456.78}
\newcommand{\defenseCouncilTitle}      % Защита, учреждение диссертационного совета
{Название учреждения}
\newcommand{\defenseCouncilAddress}    % Защита, адрес учреждение диссертационного совета
{Адрес}
\newcommand{\defenseCouncilPhone}      % Телефон для справок
{+7~(0000)~00-00-00}

\newcommand{\defenseSecretaryFio}      % Секретарь диссертационного совета, ФИО
{Фамилия Имя Отчество}
\newcommand{\defenseSecretaryRegalia}  % Секретарь диссертационного совета, регалии
{д-р~физ.-мат. наук}            % Для сокращений есть ГОСТы, например: ГОСТ Р 7.0.12-2011 + http://base.garant.ru/179724/#block_30000

\newcommand{\synopsisLibrary}          % Автореферат, название библиотеки
{Название библиотеки}
\newcommand{\synopsisDate}             % Автореферат, дата рассылки
{DD mmmmmmmm YYYY года}

% To avoid conflict with beamer class use \providecommand
\providecommand{\keywords}%            % Ключевые слова для метаданных PDF диссертации и автореферата
{}