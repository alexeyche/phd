
{\actuality} Спайковые нейронные сети являются ярким, но до сих пор не признанным в кругах машинного обучения методом. Одним из немаловажных критериев работоспособности любого метода машинного обученияя является простота модели. Этот критерий возникает не только по причине трудоемкости настраивания метода в условиях промышленной эксплуатации, но и из теоретических обоснований, которые говорят о большом риске переобучения модели. 

% {\progress} 
% Этот раздел должен быть отдельным структурным элементом по
% ГОСТ, но он, как правило, включается в описание актуальности
% темы. Нужен он отдельным структурынм элемементом или нет ---
% смотрите другие диссертации вашего совета, скорее всего не нужен.

{\aim} данной работы является найти баланс между необходимой биологической оправданностью и работоспобностью модели.

Для~достижения поставленной цели необходимо было решить следующие {\tasks}:
\begin{enumerate}
  \item Исследовать основные направления современных нейробиологических трендов в области информационной обработки сигналов нейроном и синаптической пластичности в частности.
  \item Рассмотреть свойства модели на основе обучения без учителя
  \item Исследовать применимость модели в рамках конкретных задач машинного обучения
  \item Рассмотреть свойства модели в условиях обучения с сигналом от учителя
\end{enumerate}


{\novelty}
\begin{enumerate}
  \item Впервые выведено правило обучение нейронной модели которое имеет в своей основе как и биологическую оправданность так и многолетний опыт теории нейронных сетей в виде метода обратного распространения ошибки
  \item Было выполнено оригинальное исследование рассматривающее применимость модели как на тестовых данных так и в условиях промышленной эксплуатации.
\end{enumerate}

{\influence} работы наиболее выражена в области обработки дискретных сигналов. Спайковая нейронная сеть является моделью с высокой выразительностью и методы обучения представленные в данной работе позволяют применить модель в таких задачах как: классификация и регрессионный анализ временных рядов. Метод может быть применен как в батч режиме так и в реальном времени. 

{\methods} включают в себя проведение компьютерных симуляций как на тестовых так и реальных данных.

{\defpositions}
\begin{enumerate}
  \item Выведена взаимосвязь между методом обратного распространения ошибки и биологической модели синаптической пластичности
  \item Рассмотрены факторы которые характеризуют биологический нейрон наиболее влияющие на качество модели
  \item Выявляна целесообразность применения модели в условиях промышленной эксплуатации
  \item Показано свойство масштабируемости модели
\end{enumerate}

{\reliability} полученных результатов обеспечивается рядом экспериментов. Результаты находятся в соответствии с результатами, полученными другими авторами.


{\probation}
Основные результаты работы докладывались~на: конференции нейроинформатика 2016, 2017г.

{\contribution} Автор принимал активное участие в работе со студентами на данную тематику.

%\publications\ Основные результаты по теме диссертации изложены в ХХ печатных изданиях~\cite{Sokolov,Gaidaenko,Lermontov,Management},
%Х из которых изданы в журналах, рекомендованных ВАК~\cite{Sokolov,Gaidaenko}, 
%ХХ --- в тезисах докладов~\cite{Lermontov,Management}.

\ifnumequal{\value{bibliosel}}{0}{% Встроенная реализация с загрузкой файла через движок bibtex8
    \publications\ Основные результаты по теме диссертации изложены в XX печатных изданиях, 
    X из которых изданы в журналах, рекомендованных ВАК, 
    X "--- в тезисах докладов.%
}{% Реализация пакетом biblatex через движок biber
%Сделана отдельная секция, чтобы не отображались в списке цитированных материалов
    \begin{refsection}%
        \printbibliography[heading=countauthornotvak, env=countauthornotvak, keyword=biblioauthornotvak, section=1]%
        \printbibliography[heading=countauthorvak, env=countauthorvak, keyword=biblioauthorvak, section=1]%
        \printbibliography[heading=countauthorconf, env=countauthorconf, keyword=biblioauthorconf, section=1]%
        \printbibliography[heading=countauthor, env=countauthor, keyword=biblioauthor, section=1]%
        \publications\ Основные результаты по теме диссертации изложены в \arabic{citeauthor} печатных изданиях\nocite{bib1,bib2}, 
        \arabic{citeauthorvak} из которых изданы в журналах, рекомендованных ВАК\nocite{vakbib1,vakbib2}, 
        \arabic{citeauthorconf} "--- в тезисах докладов\nocite{confbib1,confbib2}.
    \end{refsection}
}    

